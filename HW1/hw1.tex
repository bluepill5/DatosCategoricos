%%%%%%%%%%%%%%%%%%%%%%%%%%%%%%%%%%%%%%%%%%%%%%%%%%%%%%%%%%%%%%%%%%%%%%%%%%%
%  hw1.Rnw
%   
%  Autor: Alexandro Mayoral <https://github.com/bluepill5>  
% 
%%%%%%%%%%%%%%%%%%%%%%%%%%%%%%%%%%%%%%%%%%%%%%%%%%%%%%%%%%%%%%%%%%%%%%%%%%

\documentclass[a4paper]{scrartcl}\usepackage[]{graphicx}\usepackage[]{color}
%% maxwidth is the original width if it is less than linewidth
%% otherwise use linewidth (to make sure the graphics do not exceed the margin)
\makeatletter
\def\maxwidth{ %
  \ifdim\Gin@nat@width>\linewidth
    \linewidth
  \else
    \Gin@nat@width
  \fi
}
\makeatother

\definecolor{fgcolor}{rgb}{0.345, 0.345, 0.345}
\newcommand{\hlnum}[1]{\textcolor[rgb]{0.686,0.059,0.569}{#1}}%
\newcommand{\hlstr}[1]{\textcolor[rgb]{0.192,0.494,0.8}{#1}}%
\newcommand{\hlcom}[1]{\textcolor[rgb]{0.678,0.584,0.686}{\textit{#1}}}%
\newcommand{\hlopt}[1]{\textcolor[rgb]{0,0,0}{#1}}%
\newcommand{\hlstd}[1]{\textcolor[rgb]{0.345,0.345,0.345}{#1}}%
\newcommand{\hlkwa}[1]{\textcolor[rgb]{0.161,0.373,0.58}{\textbf{#1}}}%
\newcommand{\hlkwb}[1]{\textcolor[rgb]{0.69,0.353,0.396}{#1}}%
\newcommand{\hlkwc}[1]{\textcolor[rgb]{0.333,0.667,0.333}{#1}}%
\newcommand{\hlkwd}[1]{\textcolor[rgb]{0.737,0.353,0.396}{\textbf{#1}}}%

\usepackage{framed}
\makeatletter
\newenvironment{kframe}{%
 \def\at@end@of@kframe{}%
 \ifinner\ifhmode%
  \def\at@end@of@kframe{\end{minipage}}%
  \begin{minipage}{\columnwidth}%
 \fi\fi%
 \def\FrameCommand##1{\hskip\@totalleftmargin \hskip-\fboxsep
 \colorbox{shadecolor}{##1}\hskip-\fboxsep
     % There is no \\@totalrightmargin, so:
     \hskip-\linewidth \hskip-\@totalleftmargin \hskip\columnwidth}%
 \MakeFramed {\advance\hsize-\width
   \@totalleftmargin\z@ \linewidth\hsize
   \@setminipage}}%
 {\par\unskip\endMakeFramed%
 \at@end@of@kframe}
\makeatother

\definecolor{shadecolor}{rgb}{.97, .97, .97}
\definecolor{messagecolor}{rgb}{0, 0, 0}
\definecolor{warningcolor}{rgb}{1, 0, 1}
\definecolor{errorcolor}{rgb}{1, 0, 0}
\newenvironment{knitrout}{}{} % an empty environment to be redefined in TeX

\usepackage{alltt}

% Librerías necesarias
\usepackage[utf8]{inputenc}
\usepackage{amsmath}
\usepackage{amsfonts}
\usepackage{amssymb}
\usepackage{enumerate}
\usepackage{fullpage} % Maximiza el uso del espacio en la hoja.
\usepackage[left=2cm,right=2cm,top=2cm,bottom=2cm]{geometry}
\usepackage{dcolumn}
\usepackage{rotating}
\usepackage{float}
\usepackage[section]{placeins}
\usepackage{needspace}
\usepackage{hyperref}
\usepackage{setspace}
\usepackage{pifont}
\usepackage[table,xcdraw]{xcolor}
\onehalfspacing

\author{Alexandro Mayoral, ...}
\title{Tarea 1 Datos Categóricos}
\date{15/04/2015}
\IfFileExists{upquote.sty}{\usepackage{upquote}}{}
\begin{document}
\maketitle




\textbf{Problema 1}
Calcula un intervalo del 95\% de confianza para $p$, usando las gráficas del artìculo de Clopper y Pearson, para $n = 10, 20$ y apa $p.estimada = 0.3$, (es decir $x = 3, 6$) y comparala $vs.$ el intervalo usando la aproximación normal. Presenta también alguna de las opciones que te da R en la librería $binom$ ($binom.exact$, $binom.wilson$, $etc$).

\textbf{Problema 2}
Edad de primer embarazo y cáncer cervical: Para estudiar la relacion entre la edad en el primer embarazo y la aparición de cáncer cervical, se consideró un grupo de 49 mujeres con cáncercervical y 310 controles, se clasificaron según la edad en el momento de su primer embarazo: \\

\begin{table}[h]
\centering
\begin{tabular}{|
>{\columncolor[HTML]{C0C0C0}}l |l|l|}
\hline
\textbf{Edad} & \cellcolor[HTML]{C0C0C0}\textbf{Cáncer} & \cellcolor[HTML]{C0C0C0}\textbf{Control} \\ \hline
\textit{Menos de 25} & 42 & 203 \\ \hline
\textit{Más de 25} & 7 & 107 \\ \hline
\end{tabular}
\end{table}

Calcula o responde los siguientes puntos:

\begin{itemize}
  \item ¿Qué tipo de muestreo se tiene? 
\end{itemize}

\begin{itemize}
  \item Los valores esperados bajo el modelo de no asociación(¿comó lo llamarías, modelo de independencia o de homogeneida?)
\end{itemize}

\begin{itemize}
  \item ¿Qué es la correción de Yates?
\end{itemize}

\begin{itemize}
  \item Calcula la Ji cuadrada con y sin correción de Yates ¿Qué concluyes?
\end{itemize}

\begin{itemize}
  \item Muestra los valores de los residuales estandarizados
\end{itemize}

\begin{itemize}
  \item ¿Cuál es la probabilidad de desarrollar cáncer cervical?
\end{itemize}

\begin{itemize}
  \item Calcule el RR con su intervalo de confianza. Comenta o interpreta
\end{itemize}

\begin{itemize}
  \item Calcule el OR con su intervalo de confianza. Comenta o interpreta
\end{itemize}

\textbf{Problema 3}
Leer  el artículo sobre la traducción de ODDS RATIO, y hacer un resumen a lo más de una cuartilla


\textbf{Problema 4}
La siguiente tabla muestra el comportamiento de dos grupos de individuos sometidos a dos tratamientos difirentes, ¿Hay evidencias suficientes para decir que un tratmiento es mejor que otro? Utiliza la prueba exacta de Fisher. Construye las tablas $más extremas$ que la observada.\\

\begin{table}[h]
\centering
\begin{tabular}{|
>{\columncolor[HTML]{C0C0C0}}c |c|c|c|}
\hline
\textbf{Presentó mejoría} & \multicolumn{3}{c|}{\cellcolor[HTML]{C0C0C0}\textbf{Tratamiento}} \\ \hline
\textbf{} & \cellcolor[HTML]{C0C0C0}\textit{Medicamento A} & \cellcolor[HTML]{C0C0C0}\textit{Medicamento B} & \cellcolor[HTML]{C0C0C0}\textit{Total} \\ \hline
\textit{\textbf{No}} & 6 & 3 & 9 \\ \hline
\textit{\textbf{Si}} & 17 & 20 & 37 \\ \hline
\textbf{Total} & 23 & 23 & 46 \\ \hline
\end{tabular}
\end{table}


\textbf{Problema 5}
¿Es independiente la variable happiness de la variable income? Responde haciendo un análisis de correspondencias. Da los valores de la ji cuadrada y de la inercia. Interpreta el biplot.\\

\begin{table}[h]
\centering
\begin{tabular}{|
>{\columncolor[HTML]{C0C0C0}}l |c|c|c|}
\hline
\textbf{} & \multicolumn{3}{l|}{\cellcolor[HTML]{C0C0C0}\textbf{Happiness}} \\ \hline
\textbf{Income} & \multicolumn{1}{l|}{\cellcolor[HTML]{C0C0C0}\textit{\textbf{Not to happy}}} & \multicolumn{1}{l|}{\cellcolor[HTML]{C0C0C0}\textit{\textbf{Pretty happy}}} & \multicolumn{1}{l|}{\cellcolor[HTML]{C0C0C0}\textit{\textbf{Very happy}}} \\ \hline
\textit{\textbf{Above Average}} & 49 & 294 & 272 \\ \hline
\textit{\textbf{Average}} & 131 & 835 & 454 \\ \hline
\textit{\textbf{Below Average}} & 208 & 527 & 185 \\ \hline
\end{tabular}
\end{table}
 

\end{document}
